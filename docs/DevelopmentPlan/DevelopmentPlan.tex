\documentclass{article}

\usepackage{booktabs}
\usepackage{tabularx}

\title{Development Plan\\\progname}

\author{\authname}

\date{}

\input{../Comments}
%% Common Parts

\newcommand{\progname}{Software Eng} % PUT YOUR PROGRAM NAME HERE
\newcommand{\authname}{Team \#, Team Name
\\ Student 1 Matthew Collard
\\ Student 2 Sam Gorman
\\ Student 3 Ethan Kannampuzha
\\ Student 4 Kieran Gara} % AUTHOR NAMES                  

\usepackage{hyperref}
    \hypersetup{colorlinks=true, linkcolor=blue, citecolor=blue, filecolor=blue,
                urlcolor=blue, unicode=false}
    \urlstyle{same}
                                


\begin{document}

\maketitle

\begin{table}[hp]
\caption{Revision History} \label{TblRevisionHistory}
\begin{tabularx}{\textwidth}{llX}
\toprule
\textbf{Date} & \textbf{Developer(s)} & \textbf{Change}\\
\midrule
2023/09/15 & Matthew Collard & Team Member roles added\\
Date2 & Name(s) & Description of changes\\
... & ... & ...\\
\bottomrule
\end{tabularx}
\end{table}

\wss{Put your introductory blurb here.}

\section{Team Meeting Plan}

Our Plan is to have a meeting every capstone lecture that is not software eng, or is not running. When we are behind on a deliverable, we will schedule a meeting during our time between classes to meet and go over what we will need to accomplish before the due date.

\section{Team Communication Plan}
For most issues, we will communicate over Discord with each other. Matthew is in charge of communicating with the profs/supervisor. In addition, we will also have a meeting with Irene Yuan every week during Irene Yuan's Office hours. 
\newpage
\section{Team Member Roles}

\begin{tabularx}{0.8\textwidth} { 
  | >{\raggedright\arraybackslash}X 
  | >{\centering\arraybackslash}X 
  | >{\raggedleft\arraybackslash}X | }
\hline
  Role & Name & Responsibility\\
\hline
 GitHub Administrator & Sam & Merge and maintain Github Branches \\
\hline
 Final Revision Editor  & Kieran & Last editor of requirements documents, makes sure we adhere to writing guidelines  \\
\hline
Communication Director  & Matthew & Communicates with the Supervisor/Prof and any stakeholders \\
\hline
Meeting Minute Writer & Ethan & Keeps track and writes down the meeting minutes \\
\hline
Lead Developer & Kieran & Leads the development process, makes sure we are on the right track to complete our goals at the agreed upon due dates \\
\hline
Lead UI Designer & Matthew & Makes sure the user interface is clear to the user, and functioning. Implements UI based code such as buttons\\
\hline
Functional Requirement Lead & Ethan & Makes sure every function requirement is met during the coding process\\
\hline
Non-Functional Requirement Lead & Sam & Makes sure every NFR is met during the coding process\\
\hline
\end{tabularx}

\section{Workflow Plan}

\begin{itemize}
	\item For Git, we will have a master branch that will always have a working codebase, and upto date documentation. We will have a second branch called develop, this is the development branch, and all our new changes get merged into this branch. It is meant to be unstable and most of our issues will pop up in this branch. Every time develop has a stable build, we will push the code into master. We will have branches off of develop that we are going to use as our feature branches, every new line of code gets written in these branches, and when the feature is done it will get merged into develop. This double buffer system ensures we always have working code easily accessible in master. Sam will be responsible for merging develop into master. 
	\item How will you be managing issues, including template issues, issue
	classificaiton, etc.?
\end{itemize}

\section{Proof of Concept Demonstration Plan}

What is the main risk, or risks, for the success of your project?  What will you
demonstrate during your proof of concept demonstration to convince yourself that
you will be able to overcome this risk?

\section{Technology}

\begin{itemize}
\item Specific programming language
\item Specific linter tool (if appropriate)
\item Specific unit testing framework
\item Investigation of code coverage measuring tools
\item Specific plans for Continuous Integration (CI), or an explanation that CI
  is not being done
\item Specific performance measuring tools (like Valgrind), if
  appropriate
\item Libraries you will likely be using?
\item Tools you will likely be using?
\end{itemize}

\section{Coding Standard}

\section{Project Scheduling}

\wss{How will the project be scheduled?}

\end{document}